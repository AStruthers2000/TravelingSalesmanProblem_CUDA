\chapter{Traveling Salesman Problem}
\hypertarget{index}{}\label{index}\index{Traveling Salesman Problem@{Traveling Salesman Problem}}
The primary objective of this project is to explore the potential of Ant Colony Optimization (ACO), a metaheuristic inspired by the foraging behavior of ants, in solving the Travelling Salesman Problem (TSP) using Compute Unified Device Architecture (CUDA). The goal of this project is to demonstrate the feasibility and advantages of using GPGPU-\/based parallel computing in solving complex combinatorial optimization problems. By harnessing the parallel processing capabilities of CUDA, we seek to develop an efficient parallel implementation of ACO for TSP solving on GPGPUs, aiming to significantly reduce the computational time required to find near-\/optimal solutions for large-\/scale instances of the problem.

The Traveling Salesman Problem (TSP) is one of the most extensively studied combinatorial optimization problems in computer science and logistics and operations research. The classic algorithmic problem is a standard benchmark for evaluating the efficacy of various combinatorial algorithms. In this problem, a salesman is given a list of cities and must determine the shortest route that allows him to visit each city once and return to his original location. The TSP is seemingly simple in problem statement yet the complexity required in finding the optimal solution is what makes it such a studied problem. It is an NP-\/hard problem in combinatorial optimization, important in operations research and theoretical computer science. The TSP requires finding the shortest possible tour that visits each city in a network of cities exactly once while returning to the starting city. This problem and its many variants are applicable in logistics, transportation planning, and network design, among others.

Compute Unified Device Architecture (CUDA) is a parallel computing platform and application programming interface (API) model created by NVIDIA. It allows software developers to use a CUDA-\/enabled graphics processing unit (GPU) for general purpose processing, an approach known as GPGPU (General-\/\+Purpose computing on Graphics Processing Units). By utilizing CUDA, programmers can offload computationally intensive tasks from the CPU to the GPGPU, exploiting the massive parallelism inherent in GPGPU architectures to accelerate the execution of algorithms. CUDA provides a significant increase in computing performance by harnessing the power of the graphics processing unit (GPU). With millions of CUDA-\/capable GPGPUs sold to date, software developers, scientists, and researchers are finding broad-\/ranging uses for GPGPU computing with CUDA.

This project presents a unique intersection of combinatorial optimization, parallel computing, and artificial intelligence. Our primary objective is to develop a high-\/performance parallel ACO implementation capable of efficiently solving large-\/scale instances of the TSP by exploiting the computational prowess of GPGPUs. The TSP, a classic NP-\/hard problem in combinatorial optimization, challenges researchers with finding the shortest route that visits a set of cities exactly once before returning to the origin city. Traditional exact algorithms often falter when faced with the exponential explosion of possible solutions, prompting the exploration of heuristic and metaheuristic techniques like ACO. Concurrently, CUDA serves as a formidable tool for parallel computing, offering us a platform to tap into the immense parallel processing capabilities of modern GPGPUs. 